\documentclass[eng]{mgr}
\usepackage{polski}
\usepackage[utf8]{inputenc}
%\usepackage{xcolor}
\usepackage[T1]{fontenc}

%pakiety do grafiki
\usepackage{graphicx}
\usepackage{psfrag}

%Wspomaganie tabel
\usepackage{array}
%\usepackage{tabularx}
\usepackage{hhline}
\usepackage{longtable}
%Matematyka
\usepackage{amsmath}
\usepackage{amsfonts}
\usepackage{hyperref}
\usepackage{float}
%pakiet wypisujący na marginesie etykiety równań„ i rysunkóww zdefiniowanych przez \label{}, chcąc wygenerować finalną wersję dokumentu wystarczy usunąć poniższą linię
%\usepackage{showlabels}
%\newcommand{\R}{I\!\!R} %symbol liczb rzeczywistych,
%\newtheorem{theorem}{Twierdzenie}[section] %nowe otoczenie do składania twierdzenia
\usepackage{subcaption}
\usepackage{fancyref}

\newenvironment{tightitemize}{\itemize\addtolength{\itemsep}{-5pt}}{\enditemize}

\title{Zastosowanie sztucznych sieci neuronowych do diagnostyki stanów ostrego brzucha}
\engtitle{Application of artificial neural networks to the diagnosis of surgical abdomen states}
\author{Mateusz Burniak}
\supervisor{prof. dr hab. inż. Marek Kurzyński \\ Katedra Systemów i Sieci Komputerowych}
\field{Informatyka (INF)}
\specialisation{Systemy informatyki w medycynie (IMT)}


% listings
\usepackage{listings}
% Default fixed font does not support bold face
\DeclareFixedFont{\ttb}{T1}{txtt}{bx}{n}{12} % for bold
\DeclareFixedFont{\ttm}{T1}{txtt}{m}{n}{12}  % for normal

% Custom colors
\usepackage{color}
\definecolor{deepblue}{rgb}{0,0,0.5}
\definecolor{deepred}{rgb}{0.6,0,0}
\definecolor{deepgreen}{rgb}{0,0.5,0}

\newcommand\pythonstyle{\lstset{
language=Python,
basicstyle=\ttm,
otherkeywords={self},             % Add keywords here
keywordstyle=\ttb\color{deepblue},
emph={MyClass,__init__},          % Custom highlighting
emphstyle=\ttb\color{deepred},    % Custom highlighting style
stringstyle=\color{deepgreen},
frame=tb,                         % Any extra options here
showstringspaces=false            %
}}


% Python environment
\lstnewenvironment{python}[1][]
{
\pythonstyle
\lstset{#1}
}
{}

% Python for external files
\newcommand\pythonexternal[2][]{{
\pythonstyle
\lstinputlisting[#1]{#2}}}

% Python for inline
\newcommand\pythoninline[1]{{\pythonstyle\lstinline!#1!}}
