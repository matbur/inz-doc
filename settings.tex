\documentclass[eng]{mgr}
\usepackage{polski}
\usepackage[utf8]{inputenc}
%\usepackage{xcolor}
\usepackage[T1]{fontenc}

%pakiety do grafiki
\usepackage{graphicx}
\usepackage{psfrag}

%Wspomaganie tabel
\usepackage{array}
\usepackage{tabularx}
\usepackage{hhline}
%Matematyka
\usepackage{amsmath}
\usepackage{amsfonts}
\usepackage{hyperref}
\usepackage{float}
%pakiet wypisujący na marginesie etykiety równań„ i rysunkóww zdefiniowanych przez \label{}, chcąc wygenerować finalną wersję dokumentu wystarczy usunąć poniższą linię
%\usepackage{showlabels}
%\newcommand{\R}{I\!\!R} %symbol liczb rzeczywistych,
%\newtheorem{theorem}{Twierdzenie}[section] %nowe otoczenie do składania twierdzenia
\usepackage{subcaption}
\usepackage{fancyref}

\newenvironment{tightitemize}{\itemize\addtolength{\itemsep}{-5pt}}{\enditemize}

\title{Zastosowanie sztucznych sieci neuronowych do diagnostyki stanów ostrego brzucha}
\engtitle{Application of artificial neural networks to the diagnosis of surgical abdomen states}
\author{Mateusz Burniak}
\supervisor{prof. dr hab. inż. Marek Kurzyński \\ Katedra Systemów i Sieci Komputerowych}
\field{Informatyka (INF)}
\specialisation{Systemy informatyki w medycynie (IMT)}
