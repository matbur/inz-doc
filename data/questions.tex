
\begin{longtable}{|c|l|l|}
    \caption{Wszystkie cechy z wartościami}\\ \hline
    \textbf{L.p.} & \textbf{Cecha} & \textbf{Możliwe wartości} \\ \hline
    \endfirsthead
    \multicolumn{3}{c}
    {\tablename\ \thetable\ -- \textit{Wszystkie cechy z wartościami - c.d.}} \\ \hline
    \textbf{L.p.} & \textbf{Cecha} & \textbf{Możliwe wartości} \\ \hline
    \endhead
    \hline \multicolumn{3}{r}{\textit{Kontynuacja na następnej stronie}} \\
    \endfoot
    \hline
    \endlastfoot

\multicolumn{3}{|c|}{Ogólne} \\ \hline
1 & Płeć & \begin{tabular}[c]{l}1) męska  \\ 2) żeńska \end{tabular} \\ \hline
2 & Wiek & \begin{tabular}[c]{l}1) poniżej 20 lat  \\ 2) 20 - 30 lat  \\ 3) 21 - 30 lat  \\ 4) 31 - 40 lat  \\ 5) 41 - 50 lat  \\ 6) powyżej 50 lat \end{tabular} \\ \hline
\multicolumn{3}{|c|}{Ból} \\ \hline
3 & Lokalizacja bólu na początku zachorowania & \begin{tabular}[c]{l}1) prawa górna ćwiartka  \\ 2) lewa górna ćwiartka  \\ 3) górna połowa  \\ 4) prawa połowa  \\ 5) lewa połowa  \\ 6) centralny kwadrat  \\ 7) cały brzuch  \\ 8) prawa dolna ćwiartka  \\ 9) lewa dolna ćwiartka  \\ 10) dolna połowa \end{tabular} \\ \hline
4 & Lokalizacja bólu obecnie & \begin{tabular}[c]{l}0) brak bólu  \\ 1) prawa górna ćwiartka  \\ 2) lewa górna ćwiartka  \\ 3) górna połowa  \\ 4) prawa połowa  \\ 5) lewa połowa  \\ 6) centralny kwadrat  \\ 7) cały brzuch  \\ 8) prawa dolna ćwiartka  \\ 9) lewa dolna ćwiartka  \\ 10) dolna połowa \end{tabular} \\ \hline
5 & Intensywność bólu & \begin{tabular}[c]{l}0) łagodny/brak  \\ 1) umiarkowany  \\ 2) silny \end{tabular} \\ \hline
6 & Czynniki nasilające ból & \begin{tabular}[c]{l}0) brak czynników  \\ 1) oddychanie  \\ 2) kaszel  \\ 3) ruchy ciała \end{tabular} \\ \hline
7 & Czynniki przynoszące ulgę & \begin{tabular}[c]{l}0) brak czynników  \\ 1) wymioty  \\ 2) pozycja ciała \end{tabular} \\ \hline
8 & Progresja bólu & \begin{tabular}[c]{l}1) ustepujący  \\ 2) bez zmian  \\ 3) nasilający się \end{tabular} \\ \hline
9 & Czas trwania bólu & \begin{tabular}[c]{l}1) mniej niż 12 godzin  \\ 2) 12 - 24 godzin  \\ 3) 24 - 48 godzin  \\ 4) powyżej 48 godzin \end{tabular} \\ \hline
10 & Charakter bólu na początku zachorowania & \begin{tabular}[c]{l}1) przerywany  \\ 2) stały  \\ 3) kolkowy \end{tabular} \\ \hline
11 & Charakter bólu obecnie & \begin{tabular}[c]{l}0) brak bólu  \\ 1) przerywany  \\ 2) stały  \\ 3) kolkowy \end{tabular} \\ \hline
\multicolumn{3}{|c|}{Inne objawy} \\ \hline
12 & Nudności i wymioty & \begin{tabular}[c]{l}0) brak  \\ 1) nudności bez wymiotów  \\ 2) nudności z wymiotami \end{tabular} \\ \hline
13 & Apetyt & \begin{tabular}[c]{l}1) zmniejszony  \\ 2) normalny  \\ 3) zwiększony \end{tabular} \\ \hline
14 & Wypróżnienia & \begin{tabular}[c]{l}1) biegunki  \\ 2) prawidłowe  \\ 3) zaparcia \end{tabular} \\ \hline
15 & Oddawanie moczu & \begin{tabular}[c]{l}1) normalne  \\ 2) dysuria \end{tabular} \\ \hline
\multicolumn{3}{|c|}{Historia} \\ \hline
16 & Poprzednie niestrawności & \begin{tabular}[c]{l}0) nie  \\ 1) tak \end{tabular} \\ \hline
17 & Żółtaczka w przeszłości & \begin{tabular}[c]{l}0) nie  \\ 1) tak \end{tabular} \\ \hline
18 & Poprzednie operacje brzuszne & \begin{tabular}[c]{l}0) nie  \\ 1) tak \end{tabular} \\ \hline
19 & Leki & \begin{tabular}[c]{l}0) nie  \\ 1) tak \end{tabular} \\ \hline
\multicolumn{3}{|c|}{Ogólne badanie} \\ \hline
20 & Stan psychiczny & \begin{tabular}[c]{l}1) pobudzony/cierpiący  \\ 2) prawidłowy  \\ 3) apatyczny \end{tabular} \\ \hline
21 & Skóra & \begin{tabular}[c]{l}1) blada  \\ 2) prawidłowa  \\ 3) zaczerwieniona (twarz) \end{tabular} \\ \hline
22 & Temperatura (pacha) & \begin{tabular}[c]{l}1) poniżej 36.5 stC  \\ 2) 36.5 - 37 stC  \\ 3) 37 - 37.5 stC  \\ 4) 37.5 - 38 stC  \\ 5) 38 - 39 stC  \\ 6) powyżej 39 stC \end{tabular} \\ \hline
23 & Tętno & \begin{tabular}[c]{l}1) poniżej 60 /min  \\ 2) 60 - 70 /min  \\ 3) 70 - 80 /min  \\ 4) 80 - 90 /min  \\ 5) 90 - 100 /min  \\ 6) 100 - 110 /min  \\ 7) 110 - 120 /min  \\ 8) 120 - 130 /min  \\ 9) powyżej 130 /min \end{tabular} \\ \hline
\multicolumn{3}{|c|}{Oglądanie brzucha} \\ \hline
24 & Ruchy oddechowe powłok brzusznych & \begin{tabular}[c]{l}1) normalne  \\ 2) zniesione \end{tabular} \\ \hline
25 & Wzdęcia & \begin{tabular}[c]{l}0) nie  \\ 1) tak \end{tabular} \\ \hline
\multicolumn{3}{|c|}{Badania palpacyjne} \\ \hline
26 & Umiejscowienie bolesności uciskowej & \begin{tabular}[c]{l}0) brak bólu  \\ 1) prawa górna ćwiartka  \\ 2) lewa górna ćwiartka  \\ 3) górna połowa  \\ 4) prawa połowa  \\ 5) lewa połowa  \\ 6) centralny kwadrat  \\ 7) cały brzuch  \\ 8) prawa dolna ćwiartka  \\ 9) lewa dolna ćwiartka  \\ 10) dolna połowa \end{tabular} \\ \hline
27 & Objaw Blumberga & \begin{tabular}[c]{l}0) negatywny  \\ 1) pozytywny \end{tabular} \\ \hline
28 & Obrona mięśniowa & \begin{tabular}[c]{l}0) nie  \\ 1) tak \end{tabular} \\ \hline
29 & Wzmożone napięcie powłok brzusznych & \begin{tabular}[c]{l}0) nie  \\ 1) tak \end{tabular} \\ \hline
30 & Opory patologiczne & \begin{tabular}[c]{l}0) nie  \\ 1) tak \end{tabular} \\ \hline
31 & Objaw Murphy'ego & \begin{tabular}[c]{l}0) negatywny  \\ 1) pozytywny \end{tabular} \\

\end{longtable}
