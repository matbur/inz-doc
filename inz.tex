\documentclass{report}

\usepackage{listings}
\usepackage{polski}
\usepackage[utf8]{inputenc}
\usepackage[margin=1in]{geometry}
\usepackage[english,polish]{babel}\usepackage{indentfirst}
\usepackage[T1]{fontenc}
\usepackage{xcolor}

\pagecolor{black}
\color{white}

\begin{document}
    \tableofcontents

    \chapter{Wstęp}

    \section{Cel projektu}

    \section{Motywacja}

    \chapter{Problem medyczny}

    Wybrany przeze mnie problem medyczny dotyczy klasyfikacji stanów ostrego brzucha.
    Za ten stan odpowiedzialne mogą być różne choroby, które zawsze wymagają interwejcji lekarza.
    % http://www.poradnikzdrowie.pl/zdrowie/uklad-pokarmowy/ostry-brzuch-przyczyny-objawy-leczenie-ostrego-brzucha_35445.html

    \section{Opis chorób}

    Do klasyfikacji jest 8 chorób, zatem sieć neuronowa będzie miała za zadanie przypisać 1 z 8 klas.
    Są to:
    \begin{enumerate}
        \item Ostre zapalenie wyrostka robaczkowego
        \item Zapalenie uchyłków jelit
        \item Niedrożność mechaniczna jelit
        \item Perforowany wrzód trawienny
        \item Zapalenie woreczka żółciowego
        \item Ostre zapalenie trzustki
        \item Niecharakterystyczny ból brzucha
        \item Inne przyczyny ostrego bólu brzucha
    \end{enumerate}

    % histogram klas
    Histogram pokazuje, że rozkład klas jest nierównomierny.
    Na 476 obiektów aż 157 to 'Niecharakterystyczny ból brzucha' i 141 ma etykietę 'Ostre zapalenie wyrostka robaczkowego'.
    Czyli do 2 klas należy ponad 60\% obiektów.
    Może to mieć negatywny wpływ na jakość klasyfikacji.

    \section{Opis cech}

    Dane do tego problemu zawierają 31 cech.
    Są to odpowiedzi z wywiadu medycznego i wyniki przeprowadzonych badań.
    Możliwe wartości parametrów przedstawione są poniżej.
    Jak widać wszystkie liczby są naturalne mniejsze niż 11, także normalizacja czy skalowanie danych nie jest konieczne.

    \begin{itemize}
        \item Ogólne
        \begin{enumerate}
            \setcounter{enumi}{0}
            \item Płeć
            \begin{itemize}
                \item 1 - męska
                \item 2 - żeńska
            \end{itemize}
        \end{enumerate}
        \begin{enumerate}
            \setcounter{enumi}{1}
            \item Wiek
            \begin{itemize}
                \item 1 - poniżej 20 lat
                \item 2 - 20 - 30 lat
                \item 3 - 21 - 30 lat
                \item 4 - 31 - 40 lat
                \item 5 - 41 - 50 lat
                \item 6 - powyżej 50 lat
            \end{itemize}
        \end{enumerate}
        \item Ból
        \begin{enumerate}
            \setcounter{enumi}{2}
            \item Lokalizacja bólu na początku zachorowania
            \begin{itemize}
                \item 1 - prawa górna ćwiartka
                \item 2 - lewa górna ćwiartka
                \item 3 - górna połowa
                \item 4 - prawa połowa
                \item 5 - lewa połowa
                \item 6 - centralny kwadrat
                \item 7 - cały brzuch
                \item 8 - prawa dolna ćwiartka
                \item 9 - lewa dolna ćwiartka
                \item 10 - dolna połowa
            \end{itemize}
        \end{enumerate}
        \begin{enumerate}
            \setcounter{enumi}{3}
            \item Lokalizacja bólu obecnie
            \begin{itemize}
                \item 0 - brak bólu
                \item 1 - prawa górna ćwiartka
                \item 2 - lewa górna ćwiartka
                \item 3 - górna połowa
                \item 4 - prawa połowa
                \item 5 - lewa połowa
                \item 6 - centralny kwadrat
                \item 7 - cały brzuch
                \item 8 - prawa dolna ćwiartka
                \item 9 - lewa dolna ćwiartka
                \item 10 - dolna połowa
            \end{itemize}
        \end{enumerate}
        \begin{enumerate}
            \setcounter{enumi}{4}
            \item Intensywność bólu
            \begin{itemize}
                \item 0 - łagodny/brak
                \item 1 - umiarkowany
                \item 2 - silny
            \end{itemize}
        \end{enumerate}
        \begin{enumerate}
            \setcounter{enumi}{5}
            \item Czynniki nasilające ból
            \begin{itemize}
                \item 0 - brak czynników
                \item 1 - oddychanie
                \item 2 - kaszel
                \item 3 - ruchy ciała
            \end{itemize}
        \end{enumerate}
        \begin{enumerate}
            \setcounter{enumi}{6}
            \item Czynniki przynoszące ulgę
            \begin{itemize}
                \item 0 - brak czynników
                \item 1 - wymioty
                \item 2 - pozycja ciała
            \end{itemize}
        \end{enumerate}
        \begin{enumerate}
            \setcounter{enumi}{7}
            \item Progresja bólu
            \begin{itemize}
                \item 1 - ustepujący
                \item 2 - bez zmian
                \item 3 - nasilający się
            \end{itemize}
        \end{enumerate}
        \begin{enumerate}
            \setcounter{enumi}{8}
            \item Czas trwania bólu
            \begin{itemize}
                \item 1 - mniej niż 12 godzin
                \item 2 - 12 - 24 godzin
                \item 3 - 24 - 48 godzin
                \item 4 - powyżej 48 godzin
            \end{itemize}
        \end{enumerate}
        \begin{enumerate}
            \setcounter{enumi}{9}
            \item Charakter bólu na początku zachorowania
            \begin{itemize}
                \item 1 - przerywany
                \item 2 - stały
                \item 3 - kolkowy
            \end{itemize}
        \end{enumerate}
        \begin{enumerate}
            \setcounter{enumi}{10}
            \item Charakter bólu obecnie
            \begin{itemize}
                \item 0 - brak bólu
                \item 1 - przerywany
                \item 2 - stały
                \item 3 - kolkowy
            \end{itemize}
        \end{enumerate}
        \item Inne objawy
        \begin{enumerate}
            \setcounter{enumi}{11}
            \item Nudności i wymioty
            \begin{itemize}
                \item 0 - brak
                \item 1 - nudności bez wymiotów
                \item 2 - nudności z wymiotami
            \end{itemize}
        \end{enumerate}
        \begin{enumerate}
            \setcounter{enumi}{12}
            \item Apetyt
            \begin{itemize}
                \item 1 - zmniejszony
                \item 2 - normalny
                \item 3 - zwiększony
            \end{itemize}
        \end{enumerate}
        \begin{enumerate}
            \setcounter{enumi}{13}
            \item Wypróżnienia
            \begin{itemize}
                \item 1 - biegunki
                \item 2 - prawidłowe
                \item 3 - zaparcia
            \end{itemize}
        \end{enumerate}
        \begin{enumerate}
            \setcounter{enumi}{14}
            \item Oddawanie moczu
            \begin{itemize}
                \item 1 - normalne
                \item 2 - dysuria
            \end{itemize}
        \end{enumerate}
        \item Historia
        \begin{enumerate}
            \setcounter{enumi}{15}
            \item Poprzednie niestrawności
            \begin{itemize}
                \item 0 - nie
                \item 1 - tak
            \end{itemize}
        \end{enumerate}
        \begin{enumerate}
            \setcounter{enumi}{16}
            \item Żółtaczka w przeszłości
            \begin{itemize}
                \item 0 - nie
                \item 1 - tak
            \end{itemize}
        \end{enumerate}
        \begin{enumerate}
            \setcounter{enumi}{17}
            \item Poprzednie operacje brzuszne
            \begin{itemize}
                \item 0 - nie
                \item 1 - tak
            \end{itemize}
        \end{enumerate}
        \begin{enumerate}
            \setcounter{enumi}{18}
            \item Leki
            \begin{itemize}
                \item 0 - nie
                \item 1 - tak
            \end{itemize}
        \end{enumerate}
        \item Ogólne badanie
        \begin{enumerate}
            \setcounter{enumi}{19}
            \item Stan psychiczny
            \begin{itemize}
                \item 1 - pobudzony/cierpiący
                \item 2 - prawidłowy
                \item 3 - apatyczny
            \end{itemize}
        \end{enumerate}
        \begin{enumerate}
            \setcounter{enumi}{20}
            \item Skóra
            \begin{itemize}
                \item 1 - blada
                \item 2 - prawidłowa
                \item 3 - zaczerwieniona (twarz)
            \end{itemize}
        \end{enumerate}
        \begin{enumerate}
            \setcounter{enumi}{21}
            \item Temperatura (pacha)
            \begin{itemize}
                \item 1 - poniżej 36.5 stC
                \item 2 - 36.5 - 37 stC
                \item 3 - 37 - 37.5 stC
                \item 4 - 37.5 - 38 stC
                \item 5 - 38 - 39 stC
                \item 6 - powyżej 39 stC
            \end{itemize}
        \end{enumerate}
        \begin{enumerate}
            \setcounter{enumi}{22}
            \item Tętno
            \begin{itemize}
                \item 1 - poniżej 60 /min
                \item 2 - 60 - 70 /min
                \item 3 - 70 - 80 /min
                \item 4 - 80 - 90 /min
                \item 5 - 90 - 100 /min
                \item 6 - 100 - 110 /min
                \item 7 - 110 - 120 /min
                \item 8 - 120 - 130 /min
                \item 9 - powyżej 130 /min
            \end{itemize}
        \end{enumerate}
        \item Oglądanie brzucha
        \begin{enumerate}
            \setcounter{enumi}{23}
            \item Ruchy oddechowe powłok brzusznych
            \begin{itemize}
                \item 1 - normalne
                \item 2 - zniesione
            \end{itemize}
        \end{enumerate}
        \begin{enumerate}
            \setcounter{enumi}{24}
            \item Wzdęcia
            \begin{itemize}
                \item 0 - nie
                \item 1 - tak
            \end{itemize}
        \end{enumerate}
        \item Badania palpacyjne
        \begin{enumerate}
            \setcounter{enumi}{25}
            \item Umiejscowienie bolesności uciskowej
            \begin{itemize}
                \item 0 - brak bólu
                \item 1 - prawa górna ćwiartka
                \item 2 - lewa górna ćwiartka
                \item 3 - górna połowa
                \item 4 - prawa połowa
                \item 5 - lewa połowa
                \item 6 - centralny kwadrat
                \item 7 - cały brzuch
                \item 8 - prawa dolna ćwiartka
                \item 9 - lewa dolna ćwiartka
                \item 10 - dolna połowa
            \end{itemize}
        \end{enumerate}
        \begin{enumerate}
            \setcounter{enumi}{26}
            \item Objaw Blumberga
            \begin{itemize}
                \item 0 - negatywny
                \item 1 - pozytywny
            \end{itemize}
        \end{enumerate}
        \begin{enumerate}
            \setcounter{enumi}{27}
            \item Obrona mięśniowa
            \begin{itemize}
                \item 0 - nie
                \item 1 - tak
            \end{itemize}
        \end{enumerate}
        \begin{enumerate}
            \setcounter{enumi}{28}
            \item Wzmożone napięcie powłok brzusznych
            \begin{itemize}
                \item 0 - nie
                \item 1 - tak
            \end{itemize}
        \end{enumerate}
        \begin{enumerate}
            \setcounter{enumi}{29}
            \item Opory patologiczne
            \begin{itemize}
                \item 0 - nie
                \item 1 - tak
            \end{itemize}
        \end{enumerate}
        \begin{enumerate}
            \setcounter{enumi}{30}
            \item Objaw Murphy'ego
            \begin{itemize}
                \item 0 - negatywny
                \item 1 - pozytywny
            \end{itemize}
        \end{enumerate}
    \end{itemize}

    \section{Selekcja cech}

    Nie wszystkie cechy nadają się do procesu klasyfikacji, dlatego konieczne będzie przeprowadzenie selekcji cech.

    \subsection{Test chi2}

    Metoda, którą wybrałem to test chi2.

    $$
    \chi ^ 2 = \sum_{i=1}^{n}{ \frac{{(O_i - E_i) ^ 2}}{E_i}}
    $$

    \chapter{Sieć neuronowa}

    \section{Wprowadzenie}

    \section{Neuron}

    \subsection{Funkcja aktywacji}

    \section{Model wielowarstwowy}

    \chapter{Opis architektury aplikacji}

    \section{Schemat warstwy}

    \section{Schemat modelu}

    \subsection{Proces uczenia}

    \begin{lstlisting}
        class Layer:
        def __init__(self, shape, activation='sigmoid'):
        ...

        def feedforward(self, x: np.ndarray) -> np.ndarray:
        ...

        def calc_delta(self, d: np.ndarray = None):
        ...

        def calc_gradient(self):
        ...

        def update_weights(self):
        ...
    \end{lstlisting}
    \label{Schemat klasy Layer}


    Powyższy fragment kodu przedstawia schemat klasy Layer. Jest to implementacja jednej warstwy w sieci neuronowej.
    Klasa zawiera w sobie tablicę, która jest składa się z wag połączeń do poprzedniej warstwy.
    Przy tworzeniu instancji można podać funkcję aktywacji (domyślnie jest to sigmoid).



    \chapter{Przeprowadzone badania}

    \chapter{Podsumowanie}

    \section{Dalsze możliwości rozwoju}

    \section{Co mogłem zrobić lepiej}

    Tekst podsumowania

    \bibliography{./bibliography}
    \bibliographystyle{plain}

    \listoffigures
    \listoftables

\end{document}
